% !TEX root = TFG-book.tex
% !TEX spellcheck = es-ES
\chapter*{Introducción}

El análisis numérico es la rama de las matemáticas que desarrolla los algoritmos para resolver numéricamente problemas matemáticos. Donde el lenguaje simbólico presenta limitaciones o imposibilidad en su resolución, los métodos numéricos nos ofrecen una amplia gama de herramientas para desarrollar soluciones eficientes a problemas de gran importancia en multitud de campos. Algunas de las aplicaciones incluyen: "Machine Learning" (aprendizaje automático para inteligencias artificiales), redes neuronales profundas, algoritmos de buscadores web (como PageRank de Google), cálculo de trayectoria de satélites, modelos climáticos. Grandes problemas que intentan solucionarse hoy en día con los métodos numéricos pueden ser la aproximación digital de los números reales. \\


\section*{Programación numérica}

En el primer capítulo de este trabajo desarrollaremos los métodos directos de resolución de sistemas de ecuaciones no lineales, como son el método del punto fijo, el método de Newton y los métodos cuasi-Newton. Estos métodos abordan de manera directa el problema planteado mediante métodos numéricos.

En el segundo capítulo, hablaremos del método de máximo descenso y su alternativa, el método del gradiente conjugado. Estos métodos pretenden la resolución de sistemas diferenciales ordinarios transformándolos en un problema equivalente de cálculo de mínimos absolutos para nuestra función.

El tercer capítulo tratará de los métodos de continuación homotópica. Estos métodos considerarán el problema dentro de una familia, para el cual, para ciertos parámetros, podremos obtener la solución.

%\section*{Métodos abiertos y métodos de intervalos}
%
%Dentro de los métodos del cálculo numérico existen métodos como el de la Bisección, para encontrar raíces de ecuaciones no necesariamente lineales $f(x) = 0$ que requieren el pronóstico de la solución dentro de un intervalo dado por dos valores. Este tipo de métodos se denominan \textbf{métodos de intervalos}. Estos métodos son siempre convergentes, ya que están basados en la reducción del intervalo entre dos estimaciones de la raíz.\\
%Otros métodos, como el método de Newton-Raphson, requieren de un único pronóstico inicial. Este tipo de métodos se denominan \textbf{métodos abiertos}. La convergencia no está siempre asegurada para este tipo de métodos, pero, de darse, es iterativamente mucho más eficiente. Esto hace que el estudio de estos métodos se pueda considerar muy interesante en sistemas de un orden superior.
%
%%%% Ampliación de lo que hacemos en cada capítulo
%%%% Objetivo del trabajo: se trata del estudio del cálculo de ceros
%%%% Los métodos directos: Métodos iterativos que abordan de manera directa el problema planteado. 
%%%% En el 2 veremos que obtener el cero de un SDO no lineal se puede transformar en un problema equivalente de calcular los mínimos abs de una función.
%%%% En el 3, métodos que tratan de transformar el problema a partir de otro problema. Para ciertos valores paramétricos. Veremos los métodos que tratan de transformar un problema a partir de otro. Tratan de considerar el problema dentro de una familia para el cual, para ciertos valores de unos parámetros obtenemos soluciones, y a partir de esas soluciones tenemos las soluciones de nuestros problemas. 

