%!TEX root=TFG.tex
\usepackage[T1]{fontenc}
\usepackage{bussproofs}
\usepackage[utf8]{inputenc}
\usepackage[english, spanish, es-noshorthands]{babel}
\usepackage{comment}


\usepackage{hyperref}
\usepackage{url}
\usepackage{epigraph}

\usepackage{tikz}
\usetikzlibrary{babel}
\usetikzlibrary{cd}
\usepackage{minted}

%% FONTS: libertine+biolinum+stix
\usepackage{mathpazo}
\usepackage{algorithm}
\usepackage[noend]{algpseudocode}
\usepackage{svg}
\usepackage{amsmath}

\algrenewcommand\algorithmicprocedure{\textbf{procedimiento}}
\algrenewcommand\algorithmicwhile{\textbf{mientras}}
\algrenewcommand\algorithmicfor{\textbf{para}}
\algrenewcommand\algorithmicdo{\textbf{:}}
\algrenewcommand\algorithmicif{\textbf{si}}
\algrenewcommand\algorithmicthen{\textbf{entonces}}
\algrenewcommand\algorithmicreturn{\textbf{devuelve}}
\algrenewcommand\algorithmicend{\textbf{fin}}
\renewcommand\thealgorithm{}
\usepackage{verbatim}


% =====================
% = Datos importantes =
% =====================
% hay que rellenar estos datos y luego
% ir a \begin{document}
	
	\title{Métodos Numéricos de Resolución de Sistemas no Lineales}
	\author{Milagrosa Juana Sánchez Carrero}
	\date{2023}
	\newcommand{\tutores}[1]{\newcommand{\guardatutores}{#1}}
	\tutores{Prof. Tutor José Luis B. Trinidad}
	
	\makeatletter
	\AtBeginDocument{
		\hypersetup{
			pdftitle = {\@title},
			pdfauthor = {\@author},
			pdfsubject= {Matemáticas}
		}
	}
	\makeatother
	
	\newenvironment{dedication}
	{\clearpage           % we want a new page
		\thispagestyle{empty}% no header and footer
		\vspace*{\stretch{1}}% some space at the top 
		\itshape             % the text is in italics
		\raggedleft          % flush to the right margin
	}
	{\par % end the paragraph
		\vspace{\stretch{3}} % space at bottom is three times that at the top
		\clearpage           % finish off the page
	}
	
	% ======================
	% = Páginas de títulos =
	% ======================
	\makeatletter
	\edef\maintitle{\@title}
	\renewcommand\maketitle{%
		\begin{titlepage}
			\parindent=0pt
			\begin{flushleft}
				\vspace*{1.5mm}
				\setlength\baselineskip{0pt}
				\setlength\parskip{0mm}
				\begin{center}
					%\leavevmode\includegraphics[totalheight=4.5cm]{sello.pdf}
				\end{center}
			\end{flushleft}
			\vspace{1cm}
			\bgroup
			\Large \bfseries
			\begin{center}
				\@title
			\end{center}
			\egroup
			\vspace*{.5cm}
			\begin{center}
				\@author
			\end{center}
			\begin{center}
				2023
			\end{center}
			\vspace*{1.8cm}
			\begin{flushright}
				\begin{minipage}{8.45cm}
					Memoria presentada como parte de los requisitos para la obtención del título de
					Grado en Matemáticas por la Universidad de Extremadura.
					
					\vspace*{7.5mm}
					
					Tutorizada por
					% \vspace*{5mm}
				\end{minipage}\par
				\begin{tabularx}{8.45cm}[b]{@{}l}
					\guardatutores
				\end{tabularx}
			\end{flushright}
			\vspace*{\fill}
		\end{titlepage}
		%%% Esto es necesario...
		\pagestyle{tfg}
		\renewcommand{\chaptermark}[1]{\markright{\thechapter.\space ##1}}
		\renewcommand{\sectionmark}[1]{}
		\renewcommand{\subsectionmark}[1]{}
	}
	\makeatother
	
	\usepackage{minted}
	\newcommand{\newterm}[1]{\index{#1}\emph{#1}}
	\newcommand{\code}[1]{\Verb+{#1}+}
	
	% ======================================
	% = Color de la Universidad de Sevilla =
	% ======================================
	\definecolor{USred}{cmyk}{0,1.00,0.65,0.34}
	
	% =========
	% = Otros =
	% =========
	\usepackage[]{tabularx}
	\usepackage[]{enumitem}
	\setlist{noitemsep}
	
	% ==========================
	% = Matemáticas y teoremas =
	% ==========================
	\usepackage[]{amsmath}
	\usepackage[]{amsthm}
	\usepackage[]{mathtools}
	\usepackage[]{bm}
	\usepackage[]{thmtools}
	\usepackage[]{amsfonts}
	\newcommand{\marcador}{\vrule height 10pt depth 2pt width 2pt \hskip .5em\relax}
	\newcommand{\cabeceraespecial}{%
		\color{USred}%
		\normalfont\bfseries}
	\declaretheoremstyle[
	spaceabove=\medskipamount,
	spacebelow=\medskipamount,
	headfont=\cabeceraespecial\marcador,
	notefont=\cabeceraespecial,
	notebraces={(}{)},
	bodyfont=\normalfont\itshape,
	postheadspace=1em,
	numberwithin=chapter,
	headindent=0pt,
	headpunct={.}
	]{importante}
	\declaretheoremstyle[
	spaceabove=\medskipamount,
	spacebelow=\medskipamount,
	headfont=\normalfont\itshape\color{USred},
	notefont=\normalfont,
	notebraces={(}{)},
	bodyfont=\normalfont,
	postheadspace=1em,
	numberwithin=chapter,
	headindent=0pt,
	headpunct={.}
	]{normal}
	\declaretheoremstyle[
	spaceabove=\medskipamount,
	spacebelow=\medskipamount,
	headfont=\normalfont\itshape\color{USred},
	notefont=\normalfont,
	notebraces={(}{)},
	bodyfont=\normalfont,
	postheadspace=1em,
	headindent=0pt,
	headpunct={.},
	numbered=no,
	qed=\color{USred}\marcador
	]{demostracion}
	
	% Los nombres de los enunciados. Añade los que necesites.
	\declaretheorem[name=Observaci\'on, style=normal]{remark}
	\declaretheorem[name=Corolario, style=normal]{corollary}
	\declaretheorem[name=Proposici\'on, style=normal]{proposition}
	\declaretheorem[name=Lema, style=normal]{lemma}
	\declaretheorem[name=Ejemplo, style=normal]{example}
	
	\declaretheorem[name=Teorema, style=importante]{theorem}
	\declaretheorem[name=Definici\'on, style=importante]{definition}
	
	\let\proof=\undefined
	\declaretheorem[name=Demostraci\'on, style=demostracion]{proof}
	
	\usepackage{scalerel}
	\newcommand{\cat}{{\mathcal{C}}}
	\newcommand{\Set}{{Set}}
	\newcommand{\Grp}{{Grp}}
	\newcommand{\Ab}{{Ab}}
	\newcommand{\Vect}{{Vect}}
	\newcommand{\Ring}{{Ring}}
	\newcommand{\Cat}{{Cat}}
	\newcommand{\Z}{{\mathbb{Z}}}
	\newcommand{\Top}{{Top}}
	\newcommand{\Hask}{{Hask}}
	\newcommand{\sii}{{\ \Leftrightarrow\ }}
	
	\DeclareMathOperator{\dom}{dom}
	\DeclareMathOperator{\cod}{cod}
	\DeclareMathOperator{\id}{id}
	\DeclareMathOperator{\Id}{Id}
	\DeclareMathOperator{\Const}{Const}
	
	\newcommand{\underoverset}[3]{\underset{#1}{\overset{#2}{#3}}}
	
	\hyphenation{func-tor}
	\hyphenation{func-to-res}
	
	% ============================
	% = Composición de la página =
	% ============================
	
	\usepackage[papersize={\OPTpagesize},
	twoside,
	includehead,
	top=\OPTtopmargin,
	bottom=\OPTbottommargin,
	inner=\OPTinnermargin,
	outer=\OPToutermargin,
	bindingoffset=\OPTbindingoffset]{geometry}
	\linespread{1.069}
	\parskip=10pt plus 1pt minus .5pt
	\frenchspacing
	% \raggedright
	
	\setlength{\headheight}{20pt}
	% ==============================
	% = Composición de los títulos =
	% ==============================
	
	\usepackage[explicit]{titlesec}
	
	\titleformat{\chapter}[hang]
	{\Huge\sffamily\bfseries}
	{\thechapter\hspace{20pt}\textcolor{USred}{\vrule width 2pt}\hspace{20pt}}{0pt}
	{#1}
	\titleformat{\section}
	{\normalfont\Large\sffamily\bfseries}{\thesection\space\space}
	{1ex}
	{#1}
	\titleformat{\subsection}
	{\normalfont\large\sffamily}{\thesubsection\space\space}
	{1ex}
	{#1}
	
	% =======================
	% = Cabeceras de página =
	% =======================
	\usepackage[]{fancyhdr}
	\usepackage[]{emptypage}
	\fancypagestyle{tfg}{%
		\fancyhf{}%
		\renewcommand{\headrulewidth}{0pt}
		\renewcommand{\footrulewidth}{0pt}
		\fancyhead[LE]{{\normalsize\color{USred}\bfseries\thepage}\quad
			\small\textsc{\MakeLowercase{\maintitle}}}
		\fancyhead[RO]{\small\textsc{\MakeLowercase{\rightmark}}%
			\quad{\normalsize\bfseries\color{USred}\thepage}}%
	}
	\usepackage{makeidx}
	\makeindex
	\usepackage{csquotes} % comillas españolas
